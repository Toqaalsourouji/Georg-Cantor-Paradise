\documentclass[12pt, a4paper]{article}
\usepackage[utf8]{inputenc}
\usepackage{amsthm}
\usepackage{amsmath, amssymb}
\usepackage{xcolor, tikz}

\theoremstyle{plain}
\newtheorem{theorem}{Theorem}[section]
\newtheorem{corollary}[theorem]{Corollary}
\newtheorem{definition}[theorem]{Definition}
\newtheorem{example}[theorem]{Example}

\DeclareMathOperator*{\id}{id}

\parindent=0mm
\parskip=1.9mm
\linespread{1.1}

\renewcommand\qedsymbol{$\blacksquare$}
\renewcommand{\restriction}{\mathord{\upharpoonright}}

\binoppenalty=\maxdimen
\relpenalty=\maxdimen

\title{\textbf{The Paradise of Georg Cantor}}
\author{Toqa Mahmoud \\ ID: 900202324 \\ ~ \\ The American University in Cairo}
\date{Fall 2022}


\begin{document}
\maketitle

\vspace{1cm}

\begin{center}
\textit{``No one will drive us from the paradise which Cantor created for us."}\\
David Hilbert
\end{center}

\vspace{2cm}



\textbf{Abstract.}

This paper is about the mathematician Georg Cantor. The paper contains information about his education and work life and also Cantor's theorem. Cantor has devoted his life for the sake of improving mathematics and asking and answering questions. He has achieved a lot of that changed a lot in the field of mathematics like set theory.  

\newpage

\section{Cantor's Life}

In this section we will present the life story of Georg Cantor based on the article ``\textit{The Nature of Infinity}" by Jørgen Veisdal \cite{Veisdal}. 

\vspace{0.5cm}

\textbf{Introduction : } 

Cantor's father, a man with deep cultural and philosophical interests, gave Cantor lots of well-meaning advice on his life and career. Nevertheless, according to some reports, while being aware of his son's mathematical brilliance, Cantor's father would determinedly try to pressure him into engineering as a more desirable career choice than mathematics. Cantor's academic path was similar to that of many other extremely talented mathematicians in that he was recognized for his talent before the age of fifteen and he took a deep interest in his studies. Before enrolling in the Wiesbaden Gymnasium in 1860, he first attended a private school in Frankfurt called the Darmstadt nonclassical school. After graduating, Cantor enrolled at Höheren Gewerbschule in 1862 to begin his university studies. There, he studied engineering for two years before going to the Swiss Federal Polytechnic (ETH Zurich) to study mathematics. 

\vspace{0.5cm}

\textbf{Education : } 

At the University Of Berlin, Cantor was taught by Karl Weierstrass who was referred to as the "founder of modern analysis" and was a mathematician; Leopold Kronecker was a mathematician who studied algebra, logic, and, number theory; and Ernst Kummer was a mathematician and an expert in applied mathematician. 
Cantor's doctoral thesis at the University of Göttingen (1867), which was "De Aequationibus Secundi Gradus Indeterminatis" (On the Indeterminate Equations of the Second Degree). The thesis focused on a question that Carl Friedrich Gauss had not settled in his Disquisitiones Arithmeticae (1801).  

\vspace{1cm}
\textbf{Early Work: }

From 1870-1873 Cantor wrote multiple papers on a theorem named " Cantor's uniqueness theorem " which says every function from the real numbers to the real numbers can have at most one representation by a trigonometric series. In 1872, Cantor published a paper "Über die Ausdehnung eines Satzes aus der Theorie der
trignometrischen Reihen" (On the generalization of a theorem from the
theory of trigonometric series) explains that every point whose neighborhood contains an unlimited number of points from point set P is a boundary point of that set. Given that the collection of points has finite order, Cantor utilized the definition to strengthen his uniqueness theorem by demonstrating that the theory holds even if the trigonometric series diverges at an infinite number of points. 

\vspace{0.5cm}
\textbf{Set theory : }

A single work by Cantor titled Ueber eine Eigenschaft des Inbegries aller reellen algebraischen Zahlen ("On a Property of the Collection of All Real Algebraic Numbers") from 1874 is seen as the beginning of the set theory. Real numbers cannot be counted, which leads to the discovery of a distinction between numbers that belong to "the continuum" and those that belong to "a collection like the totality of real algebraic numbers," which is the fundamental and most important result it offers. Three significant findings are presented in Cantor's paper which consisted of five pages: The set of real algebraic numbers is countable, there are an unlimited number of values that are not part of any sequence in every interval [a,b], and the set of real numbers is uncountably infinite. After seventeen years Cantor finally wrote a paper in 1891 and it was named "Cantor’s Diagonal argument" which is applied to prove the uncountability of real numbers. Both proofs (1874 and 1891) reach the same conclusion: while both natural numbers and real numbers are infinite in number and thus continue indefinitely, there "aren't enough" natural numbers to create a one-to-one correspondence between them and the real numbers. In other words, Cantor's brilliant discovery demonstrated unequivocally that infinity comes in various sizes, some of which are larger than others.

\vspace{0.5cm}
\textbf{The Continuum Hypothesis : }
Cantor’s Continuum Hypothesis (CH) proves that there is no set of real numbers. Cantor spent the rest of his life attempting to provide proof that the continuum hypothesis is correct. Cantor extended the process into the transfinite because the process of taking derivatives does not necessarily end after a countably infinite number of iterations. When the strategy failed, Cantor turned to "indirect strategy," which is the main topic of his 1883 book "Foundations of a General Theory of Aggregates." The strategy was founded on his theory of powers of cardinal numbers, which introduced a class of transfinite numbers capable of counting the size of any infinite set. In this system, the continuum hypothesis would be demonstrated by determining where the power of the continuum falls on the "scale" of transfinite numbers—that it was the first non-denumerable transfinite number. Cantor spent years trying to prove (CH) but he failed and he could not find the right proof. 

\vspace{0.5cm}
\textbf{Mental Health : }

Cantor experienced his first major mental breakdown in May 1884, ten years after publishing his first proof of the uncountability of real numbers. Cantor reportedly had a nervous breakdown shortly after returning to Germany to attend to family matters. According to Arthur Schoenflies, the main cause of Cantor's distress was his resentment of the massive opposition to his work, which was championed by his former professor Leopold Kronecker in Berlin. Kronecker had been the most vocal opponent of Cantor's ideas since Cantor's 1874 paper, more than any other professional mathematician at the time. 

\vspace{0.5cm}
\textbf{Final Years : }

There is no record of Cantor being admitted to a mental hospital again after his hospitalization in 1884 until 1899. Cantor's youngest son died that year, and he reportedly lost interest in mathematics. Cantor saw Julius König's presentation of a paper in 1903 that attempted to disprove the basic tenants of transfinite set theory as a grave public humiliation. Cantor remained shaken and even briefly questioned God's existence. The incidents were followed by a series of additional hospitalizations spaced two to three years apart. He battled chronic depression for the last 20 years of his life while defending his controversial set theory and the accuracy of his proofs, mostly in response to criticism from other mathematicians in Germany. Cantor retired in 1913 after spending World War I in poverty. He returned to a mental hospital in June 1917, and on January 6, 1918, he passed away from a heart attack there.
\newpage
\section{Cantor's Theorem}




We start by defining a set.

\begin{definition}\rm
A \textit{set} is a collection of objects.
\end{definition}

\begin{example} \rm The following are examples of sets.
\end{example}
\begin{enumerate}

    \item The set of even integers less than 50 is \{2,4,6,...,44,46,48\}
    \item The set of prime numbers less than 10 is \{2,3,5,7\} 
    \item The set of multiples of 2 less than 10 \{2,4,6,8\} 
\end{enumerate}

       


\begin{definition}\rm
Let $A$ and $B$ be sets. We say that $A$ is a \textit{subset} of $B$, and write $A
\subseteq B$ if and only if every element of $A$ is also an element of $B$. 
\end{definition}

\begin{example}\rm The following are examples of subsets.

\begin{enumerate}
    \item The set of natural numbers is a subset of the set of integers, i.e., $\mathbb{N}\subseteq \mathbb{Z}$.
    \item $A$ = \{2,3,4\} is a subset of $B$ = \{2,3,4,5,6\}, ie., $\mathbf{A}\subseteq \mathbf{B}$.
    \item The set of natural numbers less than 1 is an empty set, ie., $\emptyset$
    \item The subsets of $A$ = \{1,2,3\} are $\emptyset$, \{1\}, \{2\},\{3\},\{1,2\},\{2,3\},\{1,3\},\{1,2,3\}.
\end{enumerate}
\end{example}

We next introduce the notion of a power set.  
\begin{definition}\rm
The \textit{power set} $\mathcal{P}(S)$ of a set $S$ is is the set of all subsets of $S$.
\end{definition}

\begin{example}\rm The following are examples of power sets. 
\begin{enumerate}
    \item $A$ = \{1,2\} so, $\mathcal{P}(A)$ = \{$\emptyset$, \{1\},\{2\},\{1,2\}\} 
    \item $\mathcal{P}(10, 20, 30)$ is the set of all subsets of \{10, 20, 30\}. Hence , $\mathcal{P}(10,20,30)$ = \{$\emptyset$, \{10\}, \{20\}, \{30\}, \{10,20\}, \{10,30\}, \{20,30\}, \{10, 20, 30\}.
    \item $\mathcal{P}(\{\emptyset\,\{1\}\})$ = \{\{$\emptyset\}$, \{$\emptyset\}$ , \{\{1\}\} , \{$\emptyset,\{1\}\}\}$.
\end{enumerate}
\end{example}

Let us now introduce the concept of a function and some of its properties.

\begin{definition}\rm Let $A$ and $B$ be sets.
\begin{itemize}
    \item A \textit{function} $f:A\to B$ from $A$ to $B$ is is an assignment which
for each element $ a \in A$, it assigns exactly one element $b \in B$. 
    
    \item A function $f:A\to B$ is called \textit{injective} if and only if for all $a1, a2 \in A$, we have
that if $a1 \neq a2$, then $f(a1) \neq f(a2)$.
    
    \item  A function $f:A\to B$ is called \textit{surjective} if and only if for every $b \in B$ there exists $a \in A $ such that f(a) = b. 
    
    \item  A function $f:A\to B$ is called \textit{bijective} if its both injective and surjective.
\end{itemize}
\end{definition}

Below, we explore some necessary conditions for the composition of two functions to be bijective. 

\begin{theorem}
Assume $f:A\rightarrow B$ and $g:B\rightarrow A$ are functions such that $f\circ g=\id_B$. Then $g$ is injective and $f$ is surjective.
\end{theorem}

\begin{proof} 

Let b be any element of $B$. If we set x = g (b), then 
                $f(x) = f(g(b))$, then  
                $f(x) = ($f\circ g$) $(b)$ ,then $f(x) = $\id_B$ (b)$ $, finally
                $f(x) = $b$ $.
Thus, for every b $\epsilon$ $B$ there exists an x $\epsilon$ $X$ such that $f(x) = $b$ $. Hence, $f$ is surjective. 

\end{proof}

Next, we explain how functions are used to compare the sizes of sets.

\begin{definition}\rm Let $A$ and $B$ be any sets (finite or infinite).
\begin{itemize}
    \item We say that the cardinality of $A$ is equal to the cardinality of $B$, and write $|A|=|B|$, if there exists a bijection between $A$ and $B$.
    
    \item We say that the cardinality of $A$ is less than or equal to the cardinality of $B$, and write $|A|\leq |B|$, if there exists an injective function from $A$ to $B$.
    
    \item We say that the cardinality of $A$ is strictly less than the cardinality of $B$, and write $|A|<|B|$, if there is an injective function from $A$ to $B$, but there is no bijection from $A$ to $B$. 
    
    \item A set $S$ is \textit{countably infinite} if there exists at least one bijection $f:\mathbb{N}\to S$.
\end{itemize}
\end{definition}



Observe that a countably infinite set is an infinite set for which we can enumerate \textit{all} of its elements in a sequence indexed by the natural numbers. Of course the set of natural numbers itself is a countably infinite set. We will show below another example of a countably infinite set.

\begin{theorem}
Let $ \mathcal{E}\subseteq \mathbb{Z}$ be the set of even integers and $\mathcal{O}\subseteq \mathbb{Z}$ be the set of odd integers. Show that $\mathcal{E}\times \mathcal{O}$ is countably infinite.
\end{theorem}



\begin{proof}
$\mathcal{E}$\, the set of even integers, is countably infinite because there exists a bijection between $\mathcal{E}$\ and $\mathbb{N}$, the set of natural numbers. We can present it by a formula $f(n)= {n+1}$ if n is even and $f(n)={-n}$ if n is odd.  Moreover, the $\mathcal{O}$, the set of odd integers, is countably infinite. $g(n) = {-n}$ when n is even and $g(n) = {n}$ when n is odd. By the Cartesian Product, two countably infinite sets are countably infinite. Hence, $\mathcal{E}$\ and $\mathcal{O}$ are countably infinite sets.  
\end{proof}

We will now present the main theorem of the article, Cantor's Theorem, which states that it is impossible to have a bijection between any set and its power set.



\begin{theorem}[Cantor's Theorem]
Let $Y$ be any (finite or infinite) set. [2] \cite{Roussos} Then $$|Y|<|\mathcal{P}(Y)|.$$
\end{theorem}
\begin{proof}
Let $Y$ be any arbitrary set. By definition, to show that $|Y|<|\mathcal{P}(Y)|$ we need to prove that $|Y|\leq|\mathcal{P}(Y)|$ and $|Y|\neq|\mathcal{P}(Y)|$. In other words, we need to construct an injective function from $Y$ to $\mathcal{P}(Y)$, and we need to show that it is impossible to have a bijection from $Y$ to $\mathcal{P}(Y)$.

First, we will show that $|Y|\leq |\mathcal{P}(Y)|$ by constructing an injective function $g:Y\to \mathcal{P}(Y)$. Assume set $S = {{x} | x \in Y}$. All elements in $S$ are subsets of $Y$, then $S$ \subset $\mathcal{P}(Y)$ , and    $S$   \not\subset  $\mathcal{P}(Y)|$ as  $\emptyset$ is not an element in $S$. Therefore $|Y|$ = $|S|$ \leq $|\mathcal{P}(Y)||$ . 



Second, we will prove that $|Y|\neq |\mathcal{P}(Y)|$, meaning that it is impossible to have a bijection from $Y$ to its powerset. For the sake of contradiction, assume that there is a bijection $h: Y\to \mathcal{P}(Y)$. Let C = ${ $s$\in $S$| $s$ \notin h($y$) }$. All the elements in C are elements in Y, so $ $C$ \subset $Y$ $ \Rightarrow $ $C$ \in $\mathcal {P} (Y)$ $. Since $C \in P(Y)$, $C = h(x)$ for some
$x \in Y $. There are two possibilities: $x \in c $ and $x \notin  C$ if $x \in c $, then $x \in c $ h(x) ⇒ $x \in Y $, which is a contradiction or if $x \notin  C$, then $x \notin  C$ h(x) = C, which is also a contradiction. Hence there is no bijection [3] \cite{Hammack}. 



Therefore, we have shown that $|Y| < |\mathcal{P}(Y)|$.
\end{proof}

\begin{definition}\rm
A set is \textit{uncountable} if the set can't be arranged in one-to-one correspondence with the set of positive integers.
\end{definition}

From Cantor's Theorem, we can deduce the following consequences. 



\begin{corollary}
The power set of the integers is uncountable.
\end{corollary}

\begin{proof}
By Cantor's theorem, we proved that there is no bijection between the set and its power set, hence, the power set of positive integers is uncountable.   
\end{proof} 

The following is another consequence of Cantor's Theorem.

\begin{corollary}
There are infinitely many infinite sets $A_0, A_1, A_2, A_3, \ldots$ such that for each $i\in \mathbb{N}$ we have that $|A_i|<|A_{i+1}|$ . That is, 
$$|A_0|<|A_1|<|A_2|<|A_3|< \cdots$$
In other words, there is an infinite hierarchy of infinities.
\end{corollary}
\begin{proof}

Assume that  $A_0$ = $N$  , then  $A_1$ = $\mathcal{P}(N)$ . Cantor's theorem says $|N| \leq $|\mathcal{P}(N)|. The hierarchy starts at $A_1$ = $A$ , next $A_1$ + 1 = $\mathcal{P}(A)$ = $\mathcal{P}(A_1)$ for every $a \in $N$ $. In conclusion,|$A_i$ |\leq |$A_i$ + 1|. 

\end{proof}
\vspace{4cm}

\begin{thebibliography}{9}

\bibitem{Veisdal} 
Jorgen Veisdal. \textit{The Nature of Infinity - and Beyond}. Medium, 2018.
Karl Weierstrass \textit (2022) Wikipedia. Wiki-media Foundation.
\bibitem{Roussos}
Joe Roussos. \textit{Cantor's Theorem}, 2017. 
\bibitem{Hammack}
Richard Hammack \textit{Book of Proof} Chapter 13 ( Cardinality of sets ) , 2018. 


\end{thebibliography}

\end{document}
